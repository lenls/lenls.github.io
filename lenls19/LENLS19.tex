
\documentclass[12pt]{jarticle}
\usepackage[dvips, twoside, bindingoffset=0.4cm, paper=a4paper, hmargin=1.8cm, top=3cm, bottom=1.5cm]{geometry}
\usepackage[psamsfonts]{amssymb}
\usepackage[dvips]{graphicx}

\setlength{\textheight}{24.45cm}
\setlength{\textwidth}{17.0cm}
\setlength{\evensidemargin}{-0.5cm}
\setlength{\oddsidemargin}{-0.5cm}

\begin{document}
%\thispagestyle{empty}
\pagestyle{empty}

\vspace*{1cm}
\begin{Large}
\begin{center}
{\bf Proceedings of the 19th International Workshop of \\
     Logic and Engineering of \\Natural Language Semantics 19 (LENLS19)}
\end{center}
\end{Large}
\vspace*{1cm}
\begin{Large}
\begin{center}
{\em hosted by The Association for Logic, Language and Information (FoLLI)}
\end{center}
\end{Large}
\begin{large}
\vspace*{1cm}
\begin{center}
Workshop Chair\\ $\;$\\
Daisuke Bekki (Ochanomizu University)
\end{center}
\end{large}
\vspace*{3cm}
%\begin{center}
%\ifpdf{
%\centerline{
%\raisebox{1ex}{\scalebox{.3}{\includegraphics{jsai_logo.eps}}}
%\hspace*{30pt}
%\scalebox{.3}{\includegraphics{logo05.eps}}
%}
%}
%\ifdvi{
%\centerline{
%\includegraphics[width=3.5cm,height=2cm]{jsai_logo.eps}
%}
%}
%\end{center}
\vspace*{1cm}
\begin{large}
\begin{center}
Hybrid (Ochanomizu University | Online) on 19(Sat), 20(Sun), \\
Hybrid (The University of Tokyo, Komaba 1 Campus | Online) on 21(Mon), \\
November 19(Sat), 20(Sun), and 21(Mon), 2022
\end{center}
\end{large}
\vfill

\newpage

\vspace*{20cm}
\vfill
\begin{large}

\end{large}

\newpage
\pagestyle{plain}
\pagenumbering{roman}

\newpage
\section*{Preface}
%%%%%%%%%%%%%%%%%%


This proceedings volume contains selected and invited papers on topics in formal semantics, formal pragmatics, and related fields, including the following:
\begin{itemize}
\item[$\maltese$] Formal syntax, semantics and pragmatics of natural language
\item[$\maltese$] Model-theoretic and/or proof-theoretic semantics of natural language
\item[$\maltese$] Computational approaches to semantics and pragmatics
\item[$\maltese$] Nonclassical logic and its relation to natural language (especially substructural, fuzzy, categorical, 
and topological logic)
\item[$\maltese$] Formal philosophy of language
\item[$\maltese$] Scientific methodology and/or experimental design in linguistics
\end{itemize}


LENLS is being organized by an alliance of "AI systems that can explain by language based on knowledge and reasoning" project (Grant Number JPMJCR20D2), funded by JST CREST Programs "Core technologies for trusted quality AI systems."


\subsection*{Workship Organizers/Program Committee}
\begin{flushleft}
\begin{tabular}{l}
Daisuke Bekki (Ochanomizu University) \\
Alastair Butler (Hirosaki University) \\
Patrick D. Elliott (Heinrich-Heine University of Dusseldorf) \\
Naoya Fujikawa (University of Tokyo) \\
Yurie Hara (Hokkaido University) \\
Robert Henderson (University of Arizona) \\
Hitomi Hirayama (Keio University) \\
Magdalena Kaufmann (University of Connecticut) \\
Koji Mineshima (Keio University) \\
Elin McCready (Aoyama Gakuin University) \\
Yoshiki Mori (University of Tokyo) \\
David Y. Oshima (Nagoya University) \\
Katsuhiko Sano (Hokkaido University) \\
Osamu Sawada (Kobe University) \\
Ribeka Tanaka (Ochanomizu University) \\
Wataru Uegaki (University of Edinburgh) \\
Katsuhiko Yabushita (Naruto University of Education) \\
Tomoyuki Yamada (Hokkaido University) \\
Shunsuke Yatabe (Ochanomizu University) \\
Kei Yoshimoto (Tohoku University) \\

\end{tabular}
\end{flushleft}
\newpage
  
\section*{Program and Table of Contents}

\newcommand{\slot}[2]{\noindent \underline{#1 \  #2} \\}
\newcommand{\talk}[3]{
  \noindent #2 \\ 
  \indent\indent \textit{#1} \dotfill #3 
  \smallskip \\
  }
\newcommand{\talkk}[3]{
  \noindent #2 \\ 
  \indent\indent \textit{#1}
  \smallskip \\
  }


\noindent\textbf{\large 
1st Day: November 19 (Sat)
}\\



\slot{9:50-10:00}{Opening remark}



\slot{10:00-11:30}{}
 
  
\talk{David Yoshikazu Oshima}
     {The semantic markedness of the Japanese negative preterite: Non-existence of (positive) eventualities vs. existence of negative eventualities}
     {p.1}
  
\talk{Misato Ido}
     {Semantic Properties of the Japanese Emphatic Minimizer 'NP-no-kakera' Based on the Modal Base}
     {p.6}
  
\talk{Victor Carranza Pinedo}
     {Slurs' variability, emotional dimensions and game-theoretic pragmatics}
     {p.11}



\slot{13:30-15:00}{}
 
  
\talk{Eri Tanaka and Kenta Mizutani}
     {Granularity in number and polarity effects}
     {p.16}
  
\talk{Shinnosuke Isono, Takuya Hasegawa, Kohei Kajikawa, Koichi Kono, Shiho Nakamura and Yohei Oseki}
     {Formalizing argument structures with Combinatory Categorial Grammar}
     {p.21}
  
\talk{Shinya Okano}
     {Detecting modality and evidentiality: Against purely temporal-aspectual analyses of the German semi-modal drohen}
     {p.26}



\slot{15:30-17:00}{}
 
  
\talk{Linmin Zhang}
     {Cumulative reading, QUD, and maximal informativeness}
     {p.31}
  
\talk{Liping Tang}
     {Language politeness in social network}
     {p.36}
  
\talk{Hajime Mori}
     {The Absence of NOR in Japanese}
     {p.41}



\slot{17:15-18:15}{Invited talk}

  
\talk{Michael Moortgat}
     {End-to-End Compositional Modelling of Vector-Based Semantics}
     {p.46}




\noindent\textbf{\large 
2nd Day: November 20 (Sun)
}\\



\slot{10:00-11:30}{}

  
\talk{Satoru Suzuki}
     {Measurement Theory Meets Mereology in Multidimensionality in Resemblance Nominalism}
     {p.47}
  
\talk{Yusuke Kubota and Robert Levine}
     {Extraction pathway marking as proof structure marking}
     {p.52}
  
\talk{Alastair Butler}
     {Constraining parse ambiguity with grammatical codes}
     {p.57}



\slot{13:30-15:00}{}

  
\talk{Youyou Cong}
     {In Search of a Type Theory for Fuzzy Properties}
     {p.62}
  
\talk{Akiyoshi Tomihari and Hitomi Yanaka}
     {Logical Inference System with Text-to-Image Generation for Phrase Abduction}
     {p.67}
  
\talk{Hayate Funakura}
     {Answers, Exhaustivity, and Presupposition of wh-questions in Dependent Type Semantics}
     {p.72}



\slot{15:30-17:00}{}

  
\talk{Masanobu Toyooka}
     {A Proof-Theoretic Analysis of Meaning of a Formula in a Combination of Intuitionistic and Classical Propositional Logic}
     {p.77}
  
\talk{Roussanka Loukanova}
     {Logic Operators and Quantifiers in Type-Theory of Algorithms}
     {p.82}
  
\talk{Philippe de Groote}
     {Deriving formal semantic representations from dependency structures}
     {p.87}



\slot{17:15-18:15}{Invited talk}

  
\talk{Kohei Kishida}
     {Modal Reasoning and Theorizing in Quantified Modal Logic}
     {p.92}




\noindent\textbf{\large 
3rd Day: November 21 (Mon)
}\\



\slot{10:00-11:30}{}

  
\talk{Chungmin Lee}
     {Factivity Alternation Types and Compensatory Prosodic Focus Marking}
     {p.93}
  
\talk{Kei Yoshimoto, Joseph Tabolt, Zhen Zhou, Hiromi Kaji and Tamami Shimada}
     {The Discourse Function of Aspect in Japanese}
     {p.98}
  
\talk{Masaya Taniguch and Satoshi Tojo}
     {Left-branching tree in CCG with D combinator}
     {p.103}



\slot{13:30-15:00}{}

  
\talk{Junya Fukuta and Koji Shimamura}
     {To be Canceled, or Not to be Canceled: Reconsidering the Caused Possession in the Dative Alternation Experimentally}
     {p.108}
  
\talk{Oleg Kiselyov and Haruki Watanabe}
     {Events and Relative Clauses}
     {p.113}
  
\talk{Shimpei Endo}
     {Truthmaker Semantics for Degreeism of Vagueness}
     {p.118}



\slot{15:30-17:00}{}

  
\talk{Simon Goldstein and John Hawthorne}
     {KK is Wrong Because We Say So}
     {p.121}
  
\talk{Dan Zeman}
     {Three Rich-Lexicon Theories of Slurs: A Comparison}
     {p.126}
  
\talk{David Strohmaier and Simon Wimmer}
     {Contrafactives and learnability}
     {p.130}



\slot{17:15-18:15}{Invited talk}

  
\talk{Nayuta Miki}
     {Gricean Pragmatics According to Himself}
     {p.134}



\slot{18:00-18:10}{Closing remark}



  \end{document}

